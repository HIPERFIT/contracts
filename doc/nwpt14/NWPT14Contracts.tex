\documentclass[a4paper,debug]{easychair}

\usepackage[utf8]{inputenc}
\usepackage[english]{babel}
%\usepackage{amsmath}
%\usepackage{amsfonts}
%\usepackage{amssymb}
%\usepackage{graphicx}

\usepackage{xcolor}

\newcommand{\comm}[3][red]{{\small \color{#1}{$\spadesuit$#2: #3}}}
%\renewcommand{\comm}[3][]{}
\newcommand{\jbcomment}[1]{\comm[orange]{jb}{#1}}

\begin{document}

\title{Symbolic Contract Manipulation for Foreign Exchange Options
\thanks{Work partially supported by DSF grant No.\jbcomment{yaddayadda} (\textsc{Hiperfit})}}

\titlerunning{Symb.Contract Manipulation for FX Options}

\author{Patrick Bahr
    \and
        Jost Berthold 
    \and 
        Martin Elsman 
    \and 
        Fritz Henglein\\
}

\institute{University of Copenhagen\\
    Dept. of Computer Science (DIKU)\\
    \email{\{paba,berthold,mael,henglein\}@di.ku.dk}\\
}

\authorrunning{Bahr, Berthold, Elsman, Henglein}

\clearpage
\maketitle

\jbcomment{2-3 pages of abstract}

\comm{--}{just writing up some ideas...}

Financial computations are among the domains where (embedded) Domain specific
languages (EDSLs) first came into widespread use.
The seminal work on modelling derivatives by Peyton-Jones and Eber~\cite{ICFP00SPJEber}
demonstrated the usefulness of DSLs \jbcomment{...}
Today, many banks and financial software providers use languages internally which
more or less follow their first design.

\jbcomment{
Possible structure:

shortcomings of contract DSLs in use

our approach: very simple, exploring the design space

language presentation

semantics

possible analysis and transformations (claiming them proven)

Future work: (1) more extensive analyses and (2) linking up to pricing frameworks
}


\bibliographystyle{abbrv}
\bibliography{NWPT14Contracts}

\end{document}